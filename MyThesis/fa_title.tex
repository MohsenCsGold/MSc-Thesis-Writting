% در این فایل، عنوان پایان‌نامه، مشخصات خود، متن تقدیمی‌، ستایش، سپاس‌گزاری و چکیده پایان‌نامه را به فارسی، وارد کنید.
% توجه داشته باشید که جدول حاوی مشخصات پایان‌نامه/رساله و همچنین، مشخصات داخل آن، به طور خودکار، درج می‌شود.
%%%%%%%%%%%%%%%%%%%%%%%%%%%%%%%%%%%%
% دانشگاه خود را وارد کنید
\university{ شهید بهشتی}
% دانشکده، آموزشکده و یا پژوهشکده  خود را وارد کنید
\faculty{علوم ریاضی}
% گروه آموزشی خود را وارد کنید
\degree {کارشناسی ارشد} 
% گروه آموزشی خود را وارد کنید
\subject{علوم‌کامپیوتر }
% گرایش خود را وارد کنید
\field{علوم‌کامپیوتر}
% عنوان پایان‌نامه را وارد کنید
\title{کنترل همروندی در پایگاه داده های ذاتا XML}
% نام استاد(ان) راهنما را وارد کنید
\firstsupervisor{دکتر سید کامیار ایزدی}
%\secondsupervisor{استاد راهنمای دوم}
% نام استاد(دان) مشاور را وارد کنید. چنانچه استاد مشاور ندارید، دستور پایین را غیرفعال کنید.
\firstadvisor{دکتر محمود فضلعلی}
%\secondadvisor{استاد مشاور دوم}
% نام پژوهشگر را وارد کنید
\name{محسن }
% نام خانوادگی پژوهشگر را وارد کنید
\surname{لچینانی}
% تاریخ پایان‌نامه را وارد کنید
\thesisdate{1394}
% کلمات کلیدی پایان‌نامه را وارد کنید
\keywords{کنترل همروندی ، Native XML، Query Processing}
% چکیده پایان‌نامه را وارد کنید
\fa-abstract{\noindent
در این پایان‌نامه ابتدا نمونه‌ای از منطق‌های شناختی احتمالاتی (PEL) را معرفی کرده و تمامیت آن را اثبات می‌کنیم. سپس با در نظر گرفتن مدلی ساده‌ از این منطق، در راستای تعمیم آن به منطق‌های پویا که تغییر اطلاعات در سناریوهای چند عاملی را مدل می‌کنند پیش می‌رویم.
\\
پس از توصیف مختصری از منطق‌های شناختی پویای غیر احتمالاتی، منطق شناختی پویای احتمالاتی (PDEL) را نیز با در نظر گرفتن سه گونه‌ی طبیعی احتمال، یعنی احتمال {\prior} جهان‌ها، احتمال رخداد عمل‌ها بر اساس فرایندی متناظر با دیدگاه عامل‌ها و احتمال خطا در مشاهده‌ی عمل‌ها، معرفی خواهیم کرد. این سه گونه، شیوه‌ی به‌روزرسانی تعمیم‌یافته‌ای در اختیار می‌گذارند که روشی است مناسب و طبیعی برای مدل‌سازی جریان اطلاعات.
\\
سپس برای اینکه تمامیت منطق شناختی پویای احتمالاتی را با استفاده از تمامیت منطق شناختی احتمالاتی اثبات کنیم، اصول موضوعه‌ی صحیحی ارایه می‌کنیم تا فرمول‌های شامل عملگر پویا را به فرمول‌هایی فاقد این عملگر در زبان ایستای متناظر تحویل کنند.
\\
سرانجام گونه‌ای از منطق شناختی پویای احتمالاتی، مورد نیاز برای حل معمای \lr{Monty  Hall}\!\!\! ، ارایه کرده و تمامیت آن را اثبات می‌کنیم. سپس راه‌حلی صوری برای این معما در این منطق بدست می‌آوریم. 
}
\newpage
\thispagestyle{empty}
\vtitle
\newpage
\thispagestyle{empty}
\clearpage
~~~
\newpage
\thispagestyle{empty}
\input{rights}
\newpage
\thispagestyle{empty}
\clearpage
~~~
%\newpage
%\thispagestyle{empty}
%\centerline{{\includegraphics[width=20 cm]{replyrecord}}}
%\newpage
%\thispagestyle{empty}
%\clearpage
%~~~
\newpage
 % پایان‌نامه خود را تقدیم کنید!
\begin{acknowledgementpage}

\vspace{4cm}

{\nastaliq
{\Large
تقدیم به پدر و مادر عزیزم که همواره در تمامی مراحل زندگی پشتیبان و حامی من بوده‌اند.

\vspace{1.5cm}

\newdimen\xa
\xa=\textwidth
\advance \xa by -11cm
\hspace{\xa}

}}
\end{acknowledgementpage}
\newpage
\thispagestyle{empty}
\clearpage
~~~
%%%%%%%%%%%%%%%%%%%%%%%%%%%%%%%%%%%%
\newpage
\thispagestyle{empty}
% ستایش
\baselineskip=.750cm
\ \\ \\
 
{\nastaliq
رسيدن، به دانش است و به كردار نیک...%
}\\
\vspace{.5cm}\\
{\scriptsize\nastaliq
{
 و بی دانش به كردار نيك هم نتوان رسيد، كه نيكى را پيشتر ببايد شناختن، آنگاه بجاى آوردن. پس دانش به همه حال مى‌ببايد تا به رستگارى توان رسيدن. و چون دانش راه آمد، به بهترين چيزها كه آدمى را تواند بودن. و در اوّل آفرينش حاصل نيست و بعضى از آن بى‌رنج و انديشه حاصل شود، پس هرآينه مهمتر چيزى باشد كه در حاصل كردنش عمر گذرانند، ليكن برخى هست كه بى‌انديشه حاصل آيد و بعضى را ناچار به انديشه حاجت بود، و آنچه به انديشه حاصل شود دانسته‌اى خواهد كه درو انديشه كنند تا اين نادانسته بدان انديشه كه در آن دانسته كنند دانسته شود، و از هر دانسته هر نادانسته را نتوان شناخت، بلكه هر نادانسته را به دانسته‌اى كه در خور او بود توان شناخت. و منطق آن علم است كه درو راه انداختن نادانسته به دانسته دانسته شود...
 }}
 
\vspace{.5cm}
{\nastaliq
\newdimen\xb
\xb=\textwidth
\advance \xb by -8.5cm
\hspace{\xb}
پس منطق ناگزير آمد بر جوينده‌ی رستگاری.
\RTLfootnote{مقدمه‌ی رساله‌ی منطق دانشنامه‌ی علائی، شیخ‌الرئیس ابن‌سینا}
}
\newpage
\thispagestyle{empty}
\clearpage
~~~
%%%%%%%%%%%%%%%%%%%%%%%%%%%%%%%%%%%%
\newpage
\thispagestyle{empty}
% سپاس‌گزاری
{\nastaliq
سپاس‌گزاری...
}
\\[2cm]
ستایش و سپاس مخصوص خداوندی ست که مرا در مسیر علم و دانش قرار داد.

ارزشمندترین زمان برای نیک‌اندیشان، لحظه‌ای است که در محضر استاد می نشینند و از خرمن فضلش خوشه می‌چینند. بسیار مایلم صمیمانه‌ترین قدردانی و سپاس خویش را تقدیم استاد گرامی، جناب آقای دکتر ایزدی سازم که در طول این دوره توفیق راهم شدند تا در حلقه شاگردیشان قرار گیرم و از مرتبه‌های علم و دانش ایشان بهره‌ها ببرم. همچنین از سركار خانم دکتر طهماسبی که راهنمایی‌های ایشان در انجام این پروهش همواره چراغ راهم بوده است. 



% با استفاده از دستور زیر، امضای شما، به طور خودکار، درج می‌شود
\signature 
\newpage
\thispagestyle{empty}
\clearpage
~~~
\newpage
%{\small
\abstractview
%}
\newpage
\thispagestyle{empty}
\clearpage
~~~
\newpage