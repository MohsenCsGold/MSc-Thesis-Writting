\chapter{معمای \texorpdfstring{ \lr{\textsc{Monty Hall}}}{\textsc{Monty Hall}}}
این فصل اختصاص یافته به بررسی مسأله‌ای دشوار، مشهور به معمای \lr{\textit{Monty Hall}}، با کمک منطق شناختی پویای احتمالاتی. همانطور که در بخش \ref{probabilitysource} مطرح شد تنها دو گونه از سه گونه احتمالی که معرفی شد در به‌روزرسانی در این معما کافی است و همچنین تنها عملی که در آن رخ می‌دهد اعلان عمومی است. از این رو می‌توانیم منطق را به همان دو گونه از احتمال و اعلان عمومی محدود کنیم ولی کمی از سادگی مدل‌های ایستا و پویای آن بکاهیم و بنابراین برای حل این معما یک منطق اعلان عمومی احتمالاتی را به گونه‌ای معرفی می‌کنیم که مدل‌های ایستای آن مدل‌های کریپکی احتمالاتی باشند.
\section{منطق اعلان عمومی احتمالاتی \texorpdfstring{ \lr{(PPAL)}}{(PPAL)}}
مدل‌های شناختی احتمالاتی معرفی شده در فصل 1 را به یاد بیاورید، در مدل کریپکی احتمالاتی که در اینجا معرفی می‌کنیم فضای احتمالاتی که به هر عامل $ a\in\mA $ در هر جهان $ s\in S $ تخصیص می‌دهیم به‌صورتی است که فضای نمونه‌ یعنی $ S_{a,s} $ هر زیرمجموعه‌ی دلخواهی از $ S $ می‌تواند باشد ولی $ \sigma $-جبر مجموعه‌های اندازه‌پذیر ($ \mF_{a,s} $) همواره مجموعه‌ی توانی فضای نمونه است. همچنین چون تمامی زیرمجموعه‌های تک عضوی از فضای نمونه در $ \sigma $-جبر قرار می‌گیرند، می‌توان اندازه‌ی احتمالاتی را مستقیماً روی فضای نمونه تعریف کرد و در نتیجه تعریف مدل‌های کریپکی احتمالاتی به‌صورت زیر در می‌آید:

\begin{definition}{\textbf{مدل‌های کریپکی احتمالاتی.}}\index{مدل!کریپکی!احتمالاتی}
فرض کنید $ \mA $ مجموعه‌ی عامل‌ها و $ \mbbP $ مجموعه‌ی گزاره‌های اتمی باشد. مدل کریپکی احتمالاتی ساختار $ M_{PKL}=(S,\xra{\scr{\mA}},P,V) $ است بطوریکه
\begin{itemize}
\item
$ S $ مجموعه‌ای است غیر تهی از جهان‌های ممکن،
\item
 $ \xra{\scr{\mA}} $ مجموعه‌ای است از روابط دسترسی $ \xra{a} $ که به‌ازای هر $ a\in\mA $ روی $ S $ تعریف شده‌اند،
\item
$ P:\mA\rightarrow (S\rightarrow(S_{a,s}\rightarrow[0,1])) $، تابعی احتمالاتی روی $ S_{a,s} $ به هر عامل $ a\in\mA $ و هر جهان $ s\in S $ نسبت می‌دهد (احتمالی که به $ t $ توسط تابعی که به $ a $ در $ s $ مربوط شده است نسبت داده می‌شود به‌صورت $ P_a(s)(t) $ نمایش داده می‌شود)،
\item
$ V $ به هر گزاره‌ی اتمی مجموعه‌ای از جهان‌ها نسبت می‌دهد
\end{itemize}
مجوعه‌ی همه‌ی مدل‌های کریپکی احتمالاتی را $ \mbbM_{PKL} $ می‌نامیم. برای هر زیرمجموعه‌ی $ E $ از $ S_{a,s} $ از آنجا که مجموعه‌های اندازه‌پذیر زیرمجموعه‌های $ S_{a,s} $ هستند، داریم: 
$$P_a(s)(E)=\sum_{t\in E}P_a(s)(t)$$
و برای هر فرمول $ \varphi $ در زبان تعریف می‌کنیم:
$$P_a(s)(\varphi)=\sum_{\{v\in S_{a,s}\mid M,v\vDash \varphi\}}P_a(s)(t)$$\\
\end{definition}
\begin{definition}\textbf{زبان اعلان عمومی احتمالاتی $ \mL_{PPAL} $.}\index{زبان!اعلان عمومی!احتمالاتی}
این زبان بر پایه‌ی مجموعه‌ی شمارای $ \mbbP $ از گزاره‌های اتمی، مجموعه‌ی متناهی $ \mA $ از عامل‌ها، عملگر کریپکی $ \square_a $، عملگر به‌روزرسانی $ [.] $ و نماد تابعی احتمالاتی $ \mbP_a $ شکل می‌گیرد. فرمول‌های خوش‌تعریف با استفاده از فرم \lr{Backus-Naur} به‌صورت زیر بیان می‌شوند:

\begin{equation*}
\varphi,\psi ::=\ \top\mid\bot\mid p\mid \neg\varphi\mid\varphi\land\psi\mid \square_a\varphi\mid [!\varphi]\psi\mid\sum_{i=1}^n r_i \mbP_a(\varphi_i)\geq r
\end{equation*}
که در آن $ p\in \mbbP $، $ a\in\mA $ و $ r_1,\ldots,r_n,r\in\mbbQ $. $\sum_{i=1}^n r_i \mbP_a(\varphi_i)$  یک نام از زبان $\mL_{PPAL}$ خوانده می‌شود، و $ \sum_{i=1}^n r_i \mbP_a(\varphi_i)\geq r $ فرمول $ -a $احتمالاتی از $\mL_{PPAL}$ نامیده می‌شود. قرار دهید $ \Gamma_{PPAL} $ مجموعه‌ی همه‌ی فرمول‌های خوش‌تعریف و $ P_{a,\mL_{PPAL}} $ مجموعه‌ی همه‌ی فرمول‌های $ -a $احتمالاتی از زبان $\mL_{PPAL}$ باشد، و $ T_{\mL_{PPAL}} $ را نیز مجموعه‌ی همه‌ی نام‌ها قرار دهید. اگر از این زبان عملگر به‌روزرسانی را حذف کنیم آن را $ \mL_{PKL} $ می‌نامیم.

علاوه بر همه‌ی خلاصه‌نویسی‌های پیش‌گفته‌ی قابل بیان در این زبان، از آنجا که عملی غیر از اعلان عمومی در این زبان مطرح نیست برای سادگی نوشتار از $ [\varphi]\psi $ به جای $ [!\varphi]\psi $ به‌عنوان خلاصه‌نویسی استفاده می‌کنیم.
\end{definition}
به منظور تعبیر این زبان می‌بایست به طور همزمان دو تعریف مطرح شود، یکی تعریف راستی و دیگری تعریف مدل‌های به‌روز شده. این دو تعریف به یکدیگر وابسته‌اند ولی به دور نمی‌انجامد.

\begin{definition}\label{semanticK}{\textbf{معناشناسی منطق اعلان عمومی احتمالاتی.}} \index{معناشناسی!اعلان عمومی احتمالاتی}
درستی فرمول $ \varphi\in\Gamma_{\mL_{PPAL}} $ در $ s\in S $ در مدل کریپکی احتمالاتی $ M $ با نماد $ M,s\vDash\varphi $ به‌صورت زیر تعریف می‌شود:
\\

\semanticsa{$  M,s\vDash p $}{$ s\in V(p) $}

\semanticsa{$  M,s\vDash\neg\varphi $}{$ M,s\nvDash\varphi $}

\semanticsa{$  M,s\vDash\varphi\wedge\psi $}{$  M,s\vDash\varphi $ و $  M,s\vDash\psi $}

\semanticsa{$  M,s\vDash \square_a\varphi $}{برای هر $ v\in S $, اگر $ s\xra{a}v $، آنگاه $  M,v\vDash\varphi $}

\semanticsa{$ M,s\vDash[A]\psi $}{$ M|A,s\vDash\psi $ (تعریف \ref{updatedmodel} را ببینید)}

\semanticsa{$  M,s\vDash \sum_{i=1}^n r_i \mbP_a(\varphi_i)\geq r $}{$ \sum_{i=1}^n r_i P_{a,s}(\varphi_i)\geq r $.}\\
\end{definition}
\section{معمای \texorpdfstring{ \lr{Monty Hall}}{Monty Hall}}
\begin{itemize}
\item[]\textiranic{
فرض کنید در یک مسابقه‌ی تلویزیونی شرکت کرده‌اید، و باید از میان سه در یکی را انتخاب کنید با این وصف که پشت یکی از آنها اتومبیل است و پشت دو در دیگر دوچرخه. شما دری را انتخاب می‌کنید، مثلاً در شماره 1، و مجری، که می‌داند پشت هر در چه چیزی نهفته است، دری دیگر را باز می‌کند که پشت آن دوچرخه است، مثلاً در شماره 3. او از شما می‌پرسد «آیا حاضرید دری که انتخاب کرده‌اید را با در شماره 2 عوض کنید؟». سئوال اینجاست که عوض کردن در به نفع شماست یا نه؟
}
\end{itemize}

همانطور که در \citep{Kooi2003} آمده است خانم سَوِنت\LTRfootnote{Savant} که مرتباً در کتاب گینس به‌عنوان باهوش‌ترین فرد رکورد داشته است بر این باور است که اگر انتخاب را تغییر دهید در یک سوم موارد دوچرخه می‌برید و در دو سوم موارد اتومبیل. او اینگونه استدلال می‌کند که فرض کنید شما در مرحله‌ی اول دری را انتخاب کردید که پشت آن اتومبیل است، بنابر این شما نباید انتخابتان را تغییر دهید و این در یک سوم موارد اتفاق می‌افتد. از طرف دیگر فرض کنید که انتخاب اولیه‌ی شما دری باشد که پشتش دوچرخه است، که در دو سوم موارد رخ می‌دهد. مجری نمی‌تواند دری که پشتش اتومبیل است و دری که شما انتخاب کرده‌اید را باز کند او مجبور است در دیگری را که پشتش دوچرخه است باز کند. بنابراین در حالتی که شما ابتدا دری را انتخاب کرده‌اید که پشتش دوچرخه است تغییر انتخاب، بردن ماشین را تضمین می‌کند. پس با تغییر انتخاب در دو سوم موارد برنده‌ی اتومبیل خواهید شد.

می‌خواهیم به کمک اثباتی صوری در \lr{PPAL} نشان ‌دهیم که تعویض در به نفع شماست و مدعای خانم سَوِنت را اثبات کنیم.

قبل از آن به چند لم نیازمندیم که آنها را در اینجا اثبات می‌کنیم.

\begin{lemma}\label{negP}
$ \mbP_a(\varphi)=1-\mbP_a(\neg\varphi) $
\end{lemma}
\bp
از اصل جمع‌پذیری متناهی داریم
$ \mbP_a(\top\wedge\varphi)+\mbP_a(\top\wedge\neg\varphi)=\mbP_a(\top) $
و با استفاده از احتمال راستی حکم برقرار است.
\ep
\begin{lemma}\label{identity2}
در فرمول $ \sum_{i=1}^n r_i\mbP_a(\varphi_i)\geq r $، اگر داشته باشیم   $  \mbP_a(\varphi_j)=r' $، که در آن $ 1\leq j\leq n $ و $ r' $ عددی گویا باشد، آنگاه می‌توان اثبات کرد:
$$\sum_{i=1 , i\neq j}^n r_i\mbP_a(\varphi_i)\geq r-r'$$
\end{lemma}
\bp
از فرضیات و با استفاده از خلاصه‌نویسی فرمول‌ها و اصل $\0 $-نام‌ها داریم
$$r_1\mbP_a(\varphi_1) +\cdots +r_j\mbP_a(\varphi_j) +\cdots+r_n\mbP_a(\varphi_n)\geq r$$
$$\0\mbP_a(\varphi_1)+\cdots-\mbP_a(\varphi_j) +\cdots+\0\mbP_a(\varphi_n)=-r'$$
سپس با استفاده از اصل افزودن، حکم قضیه اثبات می‌شود.
\ep
\begin{lemma}\label{identity3}
در فرمول $ [\varphi]\sum_{i=1}^n r_i\mbP_a(\varphi_i)\geq r $، اگر داشته باشیم   $  [\varphi]\mbP_a(\varphi_j)=r' $، که در آن $ 1\leq j\leq n $ و $ r' $ عددی گویا باشد، آنگاه می‌توان اثبات کرد:
$$[\varphi]\sum_{i=1}^n r_i\mbP_a(\varphi_i)\geq r\leftrightarrow[\varphi]\sum_{i=1,i\neq j}^n r_i\mbP_a(\varphi_i)\geq r-r_j r'$$
\end{lemma}
\bp
فرض کنید $ \mbP_a(\varphi)>\0 $ آنگاه هم‌ارزی‌های زیر با استفاده از اصل به‌روزرسانی احتمال 1 برقرارند:
$$[\varphi]\sum_{i=1}^n r_i\mbP_a(\varphi_i)\geq r\leftrightarrow\sum_{i=1}^n r_i\mbP_a(\varphi\wedge[\varphi]\varphi_i)\geq r\mbP_a(\varphi)$$
$$[\varphi]\mbP_a(\varphi_j)=r'\leftrightarrow\mbP_a(\varphi\wedge[\varphi]\varphi_j)=r'\mbP_a(\varphi)$$

حال با استفاده از لم \ref{identity} و خلاصه‌نویسی فرمول‌ها بدست می‌آوریم
$$[\varphi]\sum_{i=1}^n r_i\mbP_a(\varphi_i)\geq r\leftrightarrow\sum_{i=1,i\neq j}^n r_i\mbP_a(\varphi\wedge[\varphi]\varphi_i)\geq (r-r_j r')\mbP_a(\varphi)$$

و در نتیجه اصل به‌روزرسانی احتمال 1 حکم را نتیجه می‌دهد. برای حالت $ \mbP_a(\varphi)=\0 $ نیز مشابه همین استدلال با کمک اصل به‌روزرسانی احتمال 2 برقرار است.
\ep

اکنون به مدل‌سازی معما در دستگاه منطقی‌مان می‌پردازیم و سپس به اثبات مدعا روی می‌آوریم.

مجموعه‌ی عامل‌ها را $ \mA=\{c,m\} $ می‌گیریم ($ c $ معرف شرکت‌کننده و $ m $ معرف مجری) و مجموعه‌ی گزاره‌های اتمی را اجتماع سه مجموعه‌ی $ A $، $ C $ و $ O $. بطوریکه\\
$ A=\{A_1,A_2,A_3\} $ که در آن $ A_i $ یعنی اتومبیل پشت در شماره $ i $ است.\\
$ C=\{C_1,C_2,C_3\} $ که در آن $ C_i $ یعنی شرکت‌کننده ابتدا در شماره $ i $ را انتخاب کرده است.\\
$ O=\{O_1,O_2,O_3\} $ که در آن $ O_i $ یعنی در شماره $ i $ توسط مجری باز شده است.

اکنون قواعد بازی را مدل می‌کنیم. این قواعد به این شرح هستند که: تنها یک اتومبیل پشت درهاست، شرکت‌کننده تنها می‌تواند یک در را انتخاب کند و مجری تنها می‌تواند یک در را باز کند.
$$onecar\ =\oplus A,\quad onechoice\ =\oplus C,\quad oneopen\ =\oplus O$$
که $ \oplus $ یعنی «یای انحصاری\LTRfootnote{\lr{exclusive or}}». فرض می‌کنیم که شرکت‌کننده می‌بایست به اینکه اتومبیل پشت دری خاص قرار دارد احتمال $ \frac{1}{3} $ نسبت دهد. همجنین فرض می‌کنیم شرکت‌کننده با انتخاب یک در چیزی در مورد جایگاه اتومبیل کشف نمی‌کند. بنابراین شرکت‌کننده بعد از انتخاب در نیز می‌بایست این احتمال را همان $ \frac{1}{3} $ در نظر بگیرد، یعنی انتخاب شرکت‌کننده مستقل است از جایی که اتومبیل قرار دارد.
$$equal\ =\!\!\!\!\!\bigwedge_{i\in \{1,2,3\}}\!\!\!\!\!\mbP_c(A_i)=\frac{\1}{\3},\qquad\qquad independentAC\ =\!\!\!\!\!\bigwedge_{j\in\{1,2,3\}}\!\!\!\!\![C_j]equal$$

بخش اساسی بررسی این معما آن است که ببینیم تحت چه شرایطی مجری دری را باز می‌کند. او دقیقاً یک در را باز می‌کند به شرطی که شرکت‌کننده آن را انتخاب نکرده باشد و اتومبیل نیز پشت آن نباشد.
$$conditions=\!\!\!\!\!\bigwedge_{i,j\in \{1,2,3\}}[C_i](O_j\leftrightarrow(\neg A_j\wedge\neg C_j\wedge\!\!\!\bigwedge_{k\in\{1,2,3\},k\neq j}\!\!\!\!\!\!\!\!\neg O_k))$$

حال قرار دهید
$$initial=\ onecar\wedge onechoice\wedge oneopen\wedge equal\wedge independentAC\wedge conditions$$

سئوال این است که شرکت‌کننده انتخاب خود را تغییر دهد یا نه:
$$switch=\ [C_1][O_3]\mbP_c(A_1)\leq \mbP_c(A_2)$$

اگر این جمله درست باشد، احتمال اینکه شرکت‌کننده ماشین را ببرد با تغییر در انتخاب کاهش نمی‌یابد. معلوم می‌شود که $ initial $ برای بدست آوردن این نتیجه کفایت نمی‌کند. آنچه ضروری است آن است که شرکت‌کننده از برقراری شرایط اولیه‌ی بازی مطمئن باشد :$ \mbP_c(initial)=\1 $. ما همچنین به دو فرض طبیعی نیز نیاز داریم، اولاً از نظر شرکت‌کننده احتمال اینکه او در شماره 1 را انتخاب کند بزرگتر از صفر است: $ \mbP_c(C_1)>\0 $ ثانیاً بعد از اینکه شرکت‌کننده در شماره 1 را انتخاب کرد از نظر او احتمال اینکه مجری در 3 را باز کند بزرگتر از صفر است: $ [C_1]\mbP_c(O_3)>\0 $. اینها برای بدست آوردن $switch$ کفایت می‌کنند.

فرض $independentAc$ دلالت دارد بر اینکه $ [C_1]\mbP_c(A_1)=\frac{\1}{\3} $ و بنابراین:
\begin{equation}\label{a}
\mbP_c(A_1)=\mbP_c(O_3\wedge A_1)+\mbP_c(\neg O_3\wedge A_1)\ \Rightarrow\ [C_1]\mbP_c(O_3\wedge A_1)\leq\frac{\1}{\3}
\end{equation}
با در نظر گرفتن شرایط $conditions$،  $onechoice$ و $onecar$ خواهیم داشت:
\begin{equation}\label{b0}
 [C_1]\mbP_c(A_2\rightarrow O_3)=\1
\end{equation}
زیرا از شرایط $onechoice$ و $onecar$ داریم
$$C_1\leftrightarrow\neg C_1\wedge\neg C_3\quad\textrm{و}\quad A_2\leftrightarrow\neg A_1\wedge\neg A_3$$
و در نتیجه 
\begin{equation}\tag{1}
C_1\wedge A_2\rightarrow \neg C_3\wedge\neg A_3
\end{equation}
از طرف دیگر بنابر شرط $ conditions $ داریم $ [C_1]O_1\rightarrow[C_1]\neg C_1 $. از این و از $ C_1\rightarrow[C_1]C_1 $ نتیجه می‌شود $ C_1\wedge O_1\rightarrow\bot $ که معادل است با
\begin{equation}\tag{2}
\neg C_1\vee \neg O_1
\end{equation}
و به همین صورت داریم
\begin{equation}\tag{3}
\neg A_2\vee \neg O_2
\end{equation}
پس با استفاده از (1)، (2) و (3) بدست می‌آید
$$C_1\wedge A_2\rightarrow \neg C_3\wedge\neg A_3\wedge\neg O_1\wedge\neg O_2$$
و از قاعده‌ی ضرورت $ [C_1] $ داریم
$$[C_1](C_1\wedge A_2)\rightarrow [C_1](\neg C_3\wedge\neg A_3\wedge\neg O_1\wedge\neg O_2)$$
و از شرط $ conditions $ نتیجه می‌شود $ [C_1](C_1\wedge A_2)\rightarrow [C_1]O_3 $ که به خاطر اصل ثبات اتم معادل است با
$$C_1\rightarrow(A_2\rightarrow O_3)$$
در نتیجه داریم $ [C_1]C_1\rightarrow(A_2\rightarrow O_3) $ و بنابراین $ C_1\rightarrow[C_1](A_2\rightarrow O_3) $ و
$$C_1\leftrightarrow C_1\wedge[C_1](A_2\rightarrow O_3)$$
و با استفاده از قاعده‌ی هم‌ارزی نتیجه می‌شود
$$\mbP_c(C_1\wedge[C_1](A_2\rightarrow O_3))=\mbP_c(C_1)$$
و چون طبق اصل به‌روزرسانی-احتمال 1 داریم
$$\mbP_c(C_1)>\0\rightarrow([C_1]\mbP_c(A_2\rightarrow O_3)=\1\leftrightarrow \mbP_c(C_1\wedge[C_1](A_2\rightarrow O_3))=\mbP_c(C_1))$$
گزاره‌ای که به دنبالش بودیم اثبات می‌شود.
\\ \\
حال با استفاده از \ref{b0} بدست می‌آید:
\begin{equation}\label{b}
 [C_1]\mbP_c(O_3\wedge A_2)=\mbP_c(A_2) 
\end{equation}
زیرا از لم \ref{negP} می‌دانیم
$$ \mbP_c(A_2\wedge\neg O_3)=\0 $$
و از اصل جمع‌پذیری متناهی داریم
$$ \mbP_c(O_3\wedge A_2)+\mbP_c(\neg O_3\wedge A_2)=\mbP_c(A_2) $$
پس با استفاده از لم \ref{identity2} بدست می‌آید
$$ \mbP_c(O_3\wedge A_2)=\mbP_c(A_2) $$
و سپس قاعده‌ی ضرورت $ [C_1] $ حکم را اثبات می‌کند.

(\ref{b})
 به همراه $ [C_1]\mbP_c(A_2)=\frac{\1}{\3} $
(که از $independentAC$ بدست می‌آید) و با استفاده از لم \ref{identity3} ما را مجاز می‌کند که نتیجه بگیریم $ [C_1]\mbP_c(O_3\wedge A_2)=\frac{\1}{\3} $. حال خلاصه‌نویسی فرمول‌ها و لم \ref{identity3} (قرار دهید $ r_j=1 $، $ r'=\frac{\1}{\3} $ و $ r=0 $) و (\ref{a}) نتیجه دهد:
$$[C_1]\mbP_c(O_3\wedge A_1)\leq\mbP_c(O_3\wedge A_2)$$
و با استفاده از اصل ثبات اتم خواهیم داشت
\begin{equation}\label{c}
[C_1]\mbP_c(O_3\wedge[O_3]A_1)\leq\mbP_c(O_3\wedge[O_3]A_2)
\end{equation}
و در نهایت اثبات می‌شود
\begin{equation}\tag{switch}
[C_1][O_3]\mbP_c(A_1)\leq\mbP_c(A_2)
\end{equation}