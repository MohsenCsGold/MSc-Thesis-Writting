\chapter{منطق‌های شناختی پویا}
در این فصل سه منطق شناختی پویا را با این رویکرد مطرح می‌کنیم که برای هریک اصولی موسوم به اصول موضوعه‌ی {\reduction\LTRfootnote{\lr{reduction axioms}}} معرفی کرده و با اثبات صحت آنها گامی به سوی تمامیت بر می‌داریم. در انتهای فصل نیز تمامیت را در یک قضیه برای هر سه منطق اثبات خواهیم کرد.
\section{منطق‌های شناختی پویا به منظور به‌روزرسانی غیر احتمالاتی}
منطق‌های شناختی پویا جریان اطلاعات ایجاد شده توسط عمل\LTRfootnote{\lr{event}}‌ها را توصیف می‌کنند. ساده‌ترین عمل آموزنده، و نمونه‌ای رهگشا برای بیشتر این نظریه، اعلان عمومیِ\index{اعلان عمومی} گزاره‌یِ درستی چون $ A $ به گروهی از عامل‌هاست، که به‌صورت $ !A $  نمایش می‌دهیم. به‌روزرسانی برای عمل‌های پیچیده‌تر می‌تواند برحسب «مدل‌های عمل» توصیف شود، که الگوهای پیچیده‌تری از دسترسی عامل‌ها به عملِ در حال رخداد را مدل می‌کنند. پس ابتدا به‌روزرسانی منطق شناختی توسط اعلان عمومی را بررسی می‌کنیم سپس آن را به حالت کلی‌تر، برای هر نوع عمل، توسیع می‌دهیم.

\subsection{منطق اعلان عمومی \texorpdfstring{ \lr{(PAL)}}{(PAL)}}\index{منطق!اعلان عمومی}
تأثیر پویای اعلان عمومی\LTRfootnote{\lr{public announcement}} $ A $ این است که مدل (غیر احتمالاتی) جاری  $ M=(S,\sim,V) $ را به مدل به‌روز شده‌ی $ M|A $ تبدیل می‌کند. این مدل به‌روز شده با تحدید جهان‌های $ M $ به جهان‌هایی که $ A $ در آنها درست است تعریف می‌شود.

اعلان عمومی معمولاً حاوی اطلاعاتی مفید است. از این رو ممکن است که ارزش درستی عبارات شناختی در نتیجه‌ی اعلان تغییر کند. برای مثال قبل از اعلان $ A $ عامل $ a $ آن را نمی‌دانست ولی اکنون می‌داند.

\begin{definition}{\textbf{زبان اعلان عمومی.}}\index{زبان!اعلان عمومی}
زبان اعلان عمومی توسط فرم \lr{Backus-Naur} به‌صورت زیر بیان می‌شود:
\begin{equation*}
\varphi,\psi ::=\ \top\mid\bot\mid p \mid\neg\varphi\mid\varphi\wedge\psi\mid K_i\varphi\mid\left[ !\varphi\right] \psi
\end{equation*}
\end{definition}
فرمول $ [!\varphi]\psi $ به‌صورت «$ \psi $ پس از اعلان $ \varphi $ برقرار است» خوانده می‌شود. زبان بدست آمده  در مدل‌های استاندارد برای منطق شناختی نیز قابل تفسیر است. معناشناسی برای این زبان به غیر از اعلان عمومی همانند تعریف \ref{def3} می‌باشد. معناشناسی اعلان عمومی نیز به‌صورت زیر تعریف می‌شود.
\begin{definition}{\textbf{معناشناسی اعلان عمومی.}}\index{معناشناسی!اعلان عمومی}
فرض کنید مدل شناختی  $ M=(S,\sim,V) $ داده شده باشد و $ s\in S $.
\\

\semanticsb{$ M,s\vDash [!A]\varphi $}{اگر $ M,s\vDash A $ آنگاه $ M|A,s\vDash\varphi $}
\\
\\
که در آن $ M|A $ مدل $ (S',\sim ',V') $ است به طوری که، با فرض
$ \llfloor A\rrfloor =\{t\in S\mid M,t\vDash A\} $:
\begin{LTR}
\begin{itemize}
\item
$ S'=\llfloor A\rrfloor, $
\item
$ \sim'_a=\sim_a\cap(S'\times S'), $
\item
$ V'(p)=V(p)\cap S'. $
\end{itemize}
\end{LTR}
\end{definition}
اصول موضوعه‌‌ی {\reduction} در PAL به‌صورت زیر است:
\begin{align}
&[!A]p\leftrightarrow(A\rightarrow p)\label{1}\\
&[!A]\neg\varphi\leftrightarrow(A\rightarrow\neg[!A]\varphi)\label{2}\\
&[!A](\varphi\wedge\psi)\leftrightarrow([!A]\varphi\wedge[!A]\psi)\label{3}\\
&[!A]K_a \varphi\leftrightarrow(A\rightarrow K_a[!A]\varphi)\label{4}
\end{align}

\begin{theorem}\label{reduct1}\textbf{(صحت اصول موضوعه‌ی  {\reduction} برای اعلان عمومی)}
\end{theorem}
\bp
با ارجاع به هر اصل اثباتی برای آن می‌آوریم.
\begin{itemize}
\item[(\ref{1})]
\begin{align*}
M,s\vDash [!A]p &\ \ \Leftrightarrow\ \ M,s\vDash A\Rightarrow M|A,s\vDash p\tag{1}\\
&\ \ \Leftrightarrow\ \ M,s\vDash A\Rightarrow M,s\vDash p\tag{2}\\
&\ \ \Leftrightarrow\ \ M,s\vDash A\rightarrow p
\end{align*}
اگر $ M,s\vDash A $ آنگاه $ s\in S' $ و اگر $ s\in S' $  آنگاه $ V(p)=V'(p) $. در نتیجه از (1) به (2) و برعکس می‌توان رسید.
\item[(\ref{2})]
\begin{align*}
M,s\vDash[!A]\neg\varphi &\ \ \Leftrightarrow\ \ M,s\vDash A\Rightarrow M|A,s\vDash\neg \varphi\\
&\ \ \Leftrightarrow\ \ M,s\vDash A\Rightarrow (M,s\vDash A\ \textrm{و}\  M|A,s\nvDash\varphi)\\
&\ \ \Leftrightarrow\ \ M,s\vDash A\Rightarrow M,s\vDash\neg[!A]\varphi\\
&\ \ \Leftrightarrow\ \ M,s\vDash A\rightarrow\neg[!A]\varphi
\end{align*}
\end{itemize}
\ep
\subsection{منطق شناختی پویا - به‌روزرسانی مدل‌ها \texorpdfstring{ \lr{(DEL)}}{(DEL)}}\index{منطق!شناختی!پویا}
\begin{definition}{\textbf{مدل عمل\LTRfootnote{\lr{event model}}.}}\index{مدل!عمل}
فرض کنید مجموعه‌ی $ \mA $ از عامل‌ها و زبان منطقی $ \mL $ داده شده باشد، مدل عمل ساختار $ A=(E,\sim,pre) $ است بطوری که
\begin{itemize}
\item
$ E $ مجموعه‌ای متناهی و غیر تهی است از عمل‌ها،
\item
$ \sim $ مجموعه‌ای است از روابط هم‌ارزی $ \sim_a $ روی $ E $ برای هر عامل $ a\in\mA $،
\item
$ pre $ تابعی است که به هر عمل $ e\in E $ فرمولی از $ \mL $ را نسبت می‌دهد.
\end{itemize}
تابع پیش‌شرطِ\LTRfootnote{\lr{precondition function}}\index{تابع پیش‌شرط} $ pre $ با نسبت دادن فرمول $ (pre_e) $ به هر عمل در $ E $ معین می‌کند که در کدام جهان‌ها این عمل‌ها ممکن است روی دهند. این مدل‌ها را مدل به‌روزرسانی نیز می‌نامند.
\end{definition}
این مدل‌ها بسیار شبیه مدل‌های شناختی هستند، با این تفاوت که به‌جای دانش‌های مربوط به وضعیت‌های ثابت، دانش درباره‌ی عمل‌ها \index{عمل}مدل شده است.\RTLfootnote{کلمه‌ی «عمل» ترجمه‌ای است از کلمه‌ی \lr{event}، از آنجایی که این کلمه علاوه بر منطق شناختی پویا در نظریه احتمالات نیز استفاده می‌شود، باید دانست که با تفسیرهای متفاوتی در این دو مقوله به کار می‌رود. در نظریه احتمال، \lr{event} آن است که در منطق بدان گوییم گزاره. در حالی که یک \lr{event} در منطق شناختی پویا به همراه گزاره‌ی پیش‌شرط ایجاد می‌شود، ولی در واقع \lr{event}های مدلِ عمل، مدلِ شناختی داده شده را تغییر می‌دهند و خود بخشی از مدل نیستند. از این پیچیده‌تر، گاهی اوقات به تمام مدل عمل، یک \lr{event} اطلاق می‌شود.} روابط تمییز ناپذیری $ \sim $ روی عمل‌ها ابهام درباره‌ی اینکه چه عملی واقعاً رخ داده است را مدل می‌کنند. $ e\sim_a e' $ می‌تواند به این صورت خوانده شود که «اگر فرض شود که عمل $ e $ رخ داده است رخداد عمل $ e' $ با دانشِ $ a $ سازگار است». 

اعلان عمومی $ [!\varphi] $ نیز خود به نوعی یک مدل عمل است که در آن $ E=\{!\} $ و  $ \sim=\{(!,!)\}  $ و $ pre=\{(!,\varphi)\} $.

نتیجه‌ی رخداد یک عمل نمایش داده شده با $ A $ در وضعیت نمایش داده شده با $ M $ برحسب ساختاری ضربی مدل می‌شود.
\section{منطق شناختی پویای احتمالاتی\texorpdfstring{ \lr{(PDEL)}}{(PDEL)}}\index{منطق!شناختی!پویا!احتمالاتی}
برای اینکه بتوانیم به گونه‌ای صریح و شفاف در باب تغییر داده‌های احتمالاتی در قالبی شناختی-پویا استدلال کنیم، می‌بایست منطق شناختی احتمالاتی موجود را به‌وسیله‌ی اصول موضوعه‌ی {\reduction}‌ مناسب توسعه دهیم. در این بخش نشان می‌دهیم که چگونه می‌توان این کار را بر مبنای معناشناسی مدل‌های عمل احتمالاتی، که معرفی خواهد شد، انجام داد.
\begin{definition}\textbf{زبان شناختی پویای احتمالاتی.}\index{زبان!شناختی!پویا!احتمالاتی}
زبان شناختی پویای احتمالاتی به فرم \lr{Backus-Naur} به‌صورت زیر معرفی می‌شود:

\begin{equation*}
\varphi,\psi ::=\ \top\mid\bot\mid p\mid \neg\varphi\mid\varphi\land\psi\mid K_a\varphi\mid [A,e]\varphi\mid\sum_{i=1}^n r_i \mbP_a(\varphi_i)\geq r
\end{equation*}
با همان نمادگذاری منطق شناختی احتمالاتی، علاوه بر آن $ A $ مدل عمل احتمالاتی و $ e $ عملی از آن می‌باشد. فرمول‌هایی که پیش‌شرط‌ها را در مدل احتمالاتی عمل تعریف می‌کنند از همین زبانی که معرفی شد می‌آیند.

در این زبان علاوه بر خلاصه‌نویسی‌های پیش‌گفته خلاصه‌نویسی‌های زیر نیز مطرح است:

$$\langle A,e \rangle\psi :\quad \neg[A,e]\neg\psi$$
و به منظور اینکه پیش‌شرط‌ها را در یک شئ از زبان فرموله کنیم قرار می‌دهیم
\begin{equation}\label{pg0}
pre_{A,e} :\quad\bigvee_{\varphi\in\Phi,pre(\varphi,e)>0}\varphi
\end{equation}
\end{definition}
\begin{remark}\label{preAe}
در مقاله‌ی \citep{Benthem2009}، $ pre_{A,e} $ به‌صورت زیر مطرح شده است:
\begin{equation}\label{pgeq0}
pre_{A,e} :\quad\bigvee_{\varphi\in\Phi,pre(\varphi,e)\geq 0}\varphi
\end{equation}
این تعریف معادل است با $ \bigvee_{\varphi\in\Phi}\varphi $ زیرا $ pre(\varphi,e) $ تابع احتمال است و همواره بزرگتر یا مساوی صفر است.

به دلایلی که مطرح می‌شود تعریف \ref{pg0} طبیعی‌تر به نظر می‌رسد. اولاً $ pre_{A,e} $ به‌عنوان پیش‌شرط $ e  $ مطرح است پس باید شامل پیش‌شرط‌هایی باشد که به $ e $ احتمال مثبت نسبت می‌دهند، ثانیاً اگر برای هر پیش‌شرط $ \varphi $ داشته باشیم $ pre(\varphi,e)=0 $ می‌توان $ pre_{A,e} $ را تعریف کرد $ \bot $ که از دو جنبه‌ی زیر قابل دفاع است:
\begin{itemize}
\item[-]
از منظر جبری وقتی ترتیب به‌وسیله‌ی استلزام روی فرمول‌ها تعریف شده باشد داریم $ \bot=\bigvee \phi $.
\item[-]
از نقطه نظر منطقی از آنجایی که منظور ما از $ pre(\varphi,e) $ احتمال رخداد $ e $ است وقتی $ \varphi $ برقرار است، زمانی که برای هر $ \varphi\in\Phi $ داریم $ pre(\varphi,e)=0 $، $ e $ از جهت احتمالاتی امکان وقوع ندارد، بنابراین اگر ما $ pre_{A,e} $ را قرار دهیم $ \bot $ از برقراری پیش‌شرط‌های $ e $ جلوگیری به عمل آورده‌ایم و از این رو اجازه نمی‌دهیم $ e $ رخ دهد.
\end{itemize}
\end{remark}